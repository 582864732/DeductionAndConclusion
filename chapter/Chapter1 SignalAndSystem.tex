\chapter{信号与系统}

\section{离散信号的周期性质}

设有离散序列$x[n] = e^{j\omega n}$,其周期为$N$,则有
\[
    \begin{aligned}
        x[n] &= x[n+N] \\
        e^{j\omega n} &= e^{j\omega (n+N)}
    \end{aligned}
    % \label{eqa:1}
\]
要满足序列的周期性,则要满足$\omega N = 2k\pi$,
其中$k$为任意整数,即
\begin{equation}
    \frac{N}{k} = \frac{2\pi}{\omega}
    \label{eqa:离散信号的周期性质结论}
\end{equation}
其中$\frac{2\pi}{\omega}$为有理数,才能使该离散序列存在周期性。

\section{卷积}

\subsection{卷积和}

对于任意的离散序列$x[n]$,我们都可以把它分解为不同时间的加权单位脉冲的叠加,即
\begin{equation}
    \begin{aligned}
        x[n] = \cdots +x[-2]\delta[n+2]+x[-1]\delta[n+1]+x[0]\delta[n]
            +x[1]\delta[n-1]+x[2]\delta[n-2]+\cdots
    \end{aligned}
    \label{eqa:卷积和1}
\end{equation}
即
\begin{equation}
    \begin{aligned}
        x[n] = \sum_{k=-\infty}^{\infty}x[k]\delta[n-k]
    \end{aligned}
    \label{eqa:卷积和2}
\end{equation}

我们可以设系统的单位脉冲响应为$h[n]$,根据线性时不变系统的叠加性质,有
\begin{equation}
    \begin{aligned}
        y[n] = \sum_{k=-\infty}^{\infty}x[k]h[n-k]
    \end{aligned}
    \label{eqa:卷积和公式}
\end{equation}

如式\ref{eqa:卷积和公式}称为\textbf{卷积和},并且我们定义一个新符号表示\textbf{卷积和}:
\begin{equation}
    \begin{aligned}
        y[n] = x[n]*h[n]
    \end{aligned}
    \label{eqa:卷积和符号}
\end{equation}

