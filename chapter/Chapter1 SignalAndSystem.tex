\chapter{信号与系统}

\section{离散信号的周期性质}

设有离散序列$x[n] = e^{j\omega n}$,其周期为$N$,则有
\[
    \begin{aligned}
        x[n] &= x[n+N] \\
        e^{j\omega n} &= e^{j\omega (n+N)}
    \end{aligned}
    % \label{eqa:1}
\]
要满足序列的周期性,则要满足$\omega N = 2k\pi$,
其中$k$为任意整数,即
\begin{equation}
    \frac{N}{k} = \frac{2\pi}{\omega}
    \label{eqa:离散信号的周期性质结论}
\end{equation}
其中$\frac{2\pi}{\omega}$为有理数,才能使该离散序列存在周期性。

\section{卷积}

\subsection{卷积和}

对于任意的离散序列$x[n]$,我们都可以把它分解为不同时间的加权单位脉冲的叠加,即
\begin{equation}
    \begin{aligned}
        x[n] = \cdots +x[-2]\delta[n+2]+x[-1]\delta[n+1]+x[0]\delta[n]
            +x[1]\delta[n-1]+x[2]\delta[n-2]+\cdots
    \end{aligned}
    \label{eqa:卷积和1}
\end{equation}
即
\begin{equation}
    \begin{aligned}
        x[n] = \sum_{k=-\infty}^{\infty}x[k]\delta[n-k]
    \end{aligned}
    \label{eqa:卷积和2}
\end{equation}

我们可以设系统的单位脉冲响应为$h[n]$,根据线性时不变系统的叠加性质,有
\begin{equation}
    \begin{aligned}
        y[n] = \sum_{k=-\infty}^{\infty}x[k]h[n-k]
    \end{aligned}
    \label{eqa:卷积和公式}
\end{equation}

如式\ref{eqa:卷积和公式}称为\textbf{卷积和},并且我们定义一个新符号表示\textbf{卷积和}
\begin{equation}
    \begin{aligned}
        y[n] = x[n]*h[n]
    \end{aligned}
    \label{eqa:卷积和符号}
\end{equation}

\subsection{卷积积分}

与离散信号相似,我们可以认为对于任意连续信号$x(t)$都可以使用一串延时脉冲的线性组合来表示,
我们可以定义
\begin{equation}
    \delta_{\Delta}(t) = 
    \left\{
    \begin{aligned}
        &\frac{1}{\Delta},&0 \leqslant  t < \Delta\\
        &0,&Other
    \end{aligned}
    \right.
    \label{eqa:卷积积分脉冲}
\end{equation}
则信号$x(t)$可表示为
\begin{equation}
    \begin{aligned}
        x(t) = \sum_{k=-\infty}^{\infty}x(k\Delta)\delta_{\Delta}(t-k\Delta)\Delta
    \end{aligned}
    \label{eqa:卷积积分1}
\end{equation}
当$\Delta $趋于$0$,有
\begin{equation}
    \begin{aligned}
        x(t) = \lim_{\Delta\to0}\sum_{k=-\infty}^{\infty}x(k\Delta)\delta_{\Delta}(t-k\Delta)\Delta
    \end{aligned}
    \label{eqa:卷积积分2}
\end{equation}
可转化为
\begin{equation}
    \begin{aligned}
        x(t) = \int_{-\infty}^{+\infty}x(\tau)\delta(t-\tau)d\tau
    \end{aligned}
    \label{eqa:卷积积分筛选性质}
\end{equation}
对于具有单位脉冲响应$h(t)$的系统,有
\begin{equation}
    \begin{aligned}
        x(t) = \lim_{\Delta\to0}\sum_{k=-\infty}^{\infty}x(k\Delta)h_{\Delta}(t-k\Delta)\Delta
    \end{aligned}
    \label{eqa:卷积积分3}
\end{equation}
即转化成\textbf{卷积积分}
\begin{equation}
    \begin{aligned}
        y(t) = \int_{-\infty}^{+\infty}x(\tau)h(t-\tau)d\tau
    \end{aligned}
    \label{eqa:卷积积分公式}
\end{equation}
并且两个信号的卷积可表示成
\begin{equation}
    \begin{aligned}
        y(t) = x(t)*h(t)
    \end{aligned}
    \label{eqa:卷积积分符号}
\end{equation}

\section{傅里叶级数和傅里叶变换}

\subsection{特征函数}

一个信号,若系统对该信号的输出响应仅仅为一个常数乘以该信号,则称该信号为系统的\textbf{特征函数},
该常数称为系统的\textbf{特征值}。

\textbf{证明:复指数信号为线性时不变系统的特征值。}

对于输入$x(t) = e^{st}$,系统的单位脉冲响应为$h(t)$,有卷积积分
\begin{equation}
    \begin{aligned}
        y(t) &= \int_{-\infty}^{+\infty}h(\tau)x(t-\tau)d\tau\\
             &= \int_{-\infty}^{+\infty}h(\tau)e^{s(t-\tau)}d\tau\\
             &= e^{s}\int_{-\infty}^{+\infty}h(\tau)e^{-s\tau}d\tau
    \end{aligned}
    \label{eqa:傅里叶特征函数1}
\end{equation}
令
\begin{equation}
    \begin{aligned}
        H(s) = \int_{-\infty}^{+\infty}h(\tau)e^{-s\tau}d\tau
    \end{aligned}
    \label{eqa:傅里叶特征函数2}
\end{equation}
则有
\begin{equation}
    \begin{aligned}
        y(t) = H(s)x(t)
    \end{aligned}
    \label{eqa:傅里叶特征函数3}
\end{equation}

对于离散系统,同样可以证明其具有特征函数
\begin{equation}
    \begin{aligned}
        x[n] = z^{n}
    \end{aligned}
    \label{eqa:傅里叶特征函数4}
\end{equation}
与特征值
\begin{equation}
    \begin{aligned}
        H(z) = \sum_{k=-\infty}^{+\infty}h[k]z^{-k}
    \end{aligned}
    \label{eqa:傅里叶特征函数5}
\end{equation}
使得
\begin{equation}
    \begin{aligned}
        y[n] = H(z)x[n]
    \end{aligned}
    \label{eqa:傅里叶特征函数6}
\end{equation}

\subsection{连续傅里叶级数}

连续周期信号$x(t)$具有基波周期$T$,其基波频率为$\omega_{0}$,
则可将该信号用不同谐波的复指数叠加而成,表示为
\begin{equation}
    \begin{aligned}
        x(t) = \sum_{k=-\infty}^{+\infty}a_{k}e^{jk\omega_{0}t}
    \end{aligned}
    \label{eqa:傅里叶级数1}
\end{equation}
下面来求解$a_{k}$。
将式\ref{eqa:傅里叶级数1}两边同时乘$e^{-jn\omega_{0}t}$,得
\begin{equation}
    \begin{aligned}
        x(t)e^{-jn\omega_{0}t} = \sum_{k=-\infty}^{+\infty}a_{k}e^{j(k-n)\omega_{0}t}
    \end{aligned}
    \label{eqa:傅里叶级数2}
\end{equation}
两边从$0$到$T$积分得
\begin{equation}
    \begin{aligned}
        \int_{0}^{T}x(t)e^{-jn\omega_{0}t} = 
        \int_{0}^{T}\sum_{k=-\infty}^{+\infty}a_{k}e^{j(k-n)\omega_{0}t}
    \end{aligned}
    \label{eqa:傅里叶级数3}
\end{equation}
对右边交换积分次序得
\begin{equation}
    \begin{aligned}
        \int_{0}^{T}x(t)e^{-jn\omega_{0}t} = 
        \sum_{k=-\infty}^{+\infty}a_{k}[\int_{0}^{T}e^{j(k-n)\omega_{0}t}]
    \end{aligned}
    \label{eqa:傅里叶级数4}
\end{equation}
对右边积分进行分解得
\begin{equation}
    \begin{aligned}
        \int_{0}^{T}e^{j(k-n)\omega_{0}t} = 
        \int_{0}^{T}\cos[(k-n)\omega_{0}t]dt + j\int_{0}^{T}\sin[(k-n)\omega_{0}t]dt
    \end{aligned}
    \label{eqa:傅里叶级数5}
\end{equation}
我们知道$T = \frac{2\pi}{\omega_{0}}$,因此上式中三角函数得基波周期为$\frac{T}{|k-n|}$,
而$T$必定为$\frac{T}{|k-n|}$的整数倍,因此有
\begin{equation}
    \int_{0}^{T}e^{j(k-n)\omega_{0}t} = 
    \left\{
    \begin{aligned}
        T,&k=n\\
        0,&k\neq n
    \end{aligned}
    \right.
    \label{eqa:傅里叶级数6}
\end{equation}
因此式\ref{eqa:傅里叶级数4}化简为
\begin{equation}
    \begin{aligned}
        \int_{0}^{T}x(t)e^{-jn\omega_{0}t} = Ta_{n}
    \end{aligned}
    \label{eqa:傅里叶级数7}
\end{equation}
因此有
\begin{equation}
    \begin{aligned}
        a_{n} = \frac{1}{T}\int_{0}^{T}x(t)e^{-jn\omega_{0}t} 
    \end{aligned}
    \label{eqa:傅里叶级数8}
\end{equation}